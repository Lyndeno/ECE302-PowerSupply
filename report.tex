\documentclass[12pt]{article}

\usepackage[margin=0.75in]{geometry}

\title{ {\Huge Dual Rail Power Supply } \\
    ECE 302 Lab \#1 \\ LAD D11 / Bench \#9}
\author{
    Lyndon Sanche\\
    \texttt{lsanche@ualberta.ca}
    \and
    David Lenfesty\\
    \texttt{lenfesty@ualberta.ca}
}

\begin{document}
\pagenumbering{roman}

\maketitle
\newpage
\tableofcontents

\newpage
\pagenumbering{arabic}

\section{Abstract}

Low-noise, low power split rail power supplies are a staple of the audio industry.
In order to generate the required audio waveforms for applications such as headphone or
low-power amplifiers, a negative and a positive voltage rail is necessary.

In this lab, a $\pm 10 V$ split rail power supply was implemented,
where the output ripple was designed to be less than $ 0.5 \% $ of the
regulated voltage. This supply operated using mains $115 V_RMS, 60Hz $ power, with
a step down transformer leading to a diode rectifier. This rectifier supplied DC power to
positive and negative rails, each using a capacitor as a low-pass filter and implementing
a regulator output. The positive rail used an off the shelf 78L10AC linear regulator,
while the negative rail used a Zener diode-based regulator circuit.

\section{Design}
The purpose of this power supply is to take a 60Hz, 120Vrms signal, and convert it to a +-10V DC signal. The circuit is divided into
several stages for easier understanding. These stages include a transformer stage, rectifier stage, filter stage, and regulator stage. The combination 
of these stages allow us to convert a 120Vrms signal from a standard wall outlet, and convert it into two rails, one supplying roughly 10V and one supplying -10V. 
The supply is designed to have a minimal amount of ripple so the current going into the load does not flucuate.


\section{Simulation}

\section{Results}

Implementation was straightforward, where the theoretical, simulated circuit was simply
copied onto a solderless breadboard. 

\section{Conclusion}


\end{document}
