\documentclass{article}

\title{Dual Rail Power Supply}
\author{
    Lyndon Sanche\\
    \texttt{lsanche@ualberta.ca}
    \and
    David Lenfesty\\
    \texttt{lenfesty@ualberta.ca}
}

\begin{document}
\maketitle
\newpage
\tableofcontents

\section{Introduction}

\section{Design}
The purpose of this power supply is to take a 60Hz, 120Vrms signal, and convert it to a +-10V DC sige thagnal. The circuit is divided into
several stages for easier understanding. These stages include a transformer stage, rectifier stage, filter stage, and regulator stage. The combination 
of these stages allow us to convert a 120Vrms signal from a standard wall outlet, and convert it into two rails, one supplying roughly 10V and one supplying -10V. 
The supply is designed to have a minimal amount of ripple so the current going into the load does not flucuate.

The transformer stage is used to step down the 120Vrms voltage down to a voltage that will be manageable by the rectifier, filter, and regulator stages.

The rectiifer stage is the first part of converting our signal to positive and negative rails. We have chosen to use a bridge recitifier design as it converts all parts of the voltage signal 
to be positive or negative, depending on the rail. We have chosen the bridge rectifier instead of a rectifier design such as a half wave recitifer, as it doesn't cut out the negative parts of our wave, flipping them instead 
so we have more consistent voltage in our supply. We hook up the positive and negative rails to the same recitifer to produce a positive and negative recitified signal.


\section{Simulation}

\section{Implementation}

\section{Testing}


\end{document}
