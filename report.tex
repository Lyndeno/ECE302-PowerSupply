\documentclass[12pt]{article}

\usepackage[margin=0.75in]{geometry}

\title{ {\Huge Dual Rail Power Supply } \\
    ECE 302 Lab \#1 \\ LAD D11 / Bench \#9}
\author{
    Lyndon Sanche\\
    \texttt{lsanche@ualberta.ca}
    \and
    David Lenfesty\\
    \texttt{lenfesty@ualberta.ca}
}

\begin{document}
\pagenumbering{roman}

\maketitle


\section{Abstract}

Low-noise, low power split rail power supplies are a staple of the audio industry.
In order to generate the required audio waveforms for applications such as headphone or
low-power amplifiers, a negative and a positive voltage rail is necessary.

In this lab, a $\pm 10 V$ split rail power supply was implemented,
where the output ripple was designed to be less than $ 0.5 \% $ of the
regulated voltage. This supply operated using mains $115 V_RMS, 60Hz $ power, with
a step down transformer leading to a diode rectifier. This rectifier supplied DC power to
positive and negative rails, each using a capacitor as a low-pass filter and implementing
a regulator output. The positive rail used an off the shelf 78L10AC linear regulator,
while the negative rail used a Zener diode-based regulator circuit.

\newpage
\pagenumbering{arabic}

\section{Objectives}

The objective of this lab is to design a split-rail power supply capable of supplying
$\pm 10V$ at 25mA, with a voltage error of less than 5\% and less than 0.5\% ripple.

This power supply will be used later in this lab to provide power for the audio amplifier
we will design.

\section{Design}

In order to simplify the design process, the power supply design was split into three
stages, rectification, filtering, and regulation. By splitting up the design into these
parts, it was simpler to test and easier to diagnose issues.

By simulating the circuit prior to building it, we were able to experiment wth different
component values and the effects of removing certain key components, which gave a better
understanding of how the circuit worked.

For the rectifier stage, a full-bridge rectifier was chosen as the design, as it provided
the most efficient and uniform output, compared to the half- or full-wave rectifier designs.

The largest size capacitor available in the lab kit was the one that was eventually chosen,
at $100 \mu F$, it provided plenty of filtering and smoothed out the rectifier waveform
well enough to limit ripple significantly.

The final circuit schematic can be found in appendix oiawdwiadnawodn

\section{Simulation}

\section{Results}

Implementation was straightforward, where the theoretical, simulated circuit was simply
copied onto a solderless breadboard. 

\section{Discussion}

The zener diode typically performs

\section{Conclusion}



\end{document}
